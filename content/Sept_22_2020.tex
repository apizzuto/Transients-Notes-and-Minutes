\chapter{September 22, 2020}
We discussed moving our Tuesday timeslot, but didn't come to any conclusions.

\section{Alex: self-triggered alert followup}
I showed some updated luminosity / density constraints as well as a new fast response flux sensitivity plot (not time-integrated flux)

\section{Raamis: GW status}
Raamis showed his differential upper limit calculation.

Naoko asked why doesn't raamis just fix the index in the optical follow up because there is bias in the spectral fit. Assuming an $E^{-2}$ seems to be almost the same as floating the index.

Raamis also looked into what was weird with the double bump TS structure with the spatial prior analysis in GRECO, Raamis said he will discuss this with Aswathi offline

\section{Abhishek: MOJAVE and Fast Response}
Abhishek added a line with a completeness factor to the MOJAVE differential sensitivity. We discussed whether or not there is a physical and not observational definition of completeness for this sample.

Abhishek has begun running the fast response code as well. We discussed writing a latency page similar to the GW latency page that we had.

\section{Ibrahim: GZK and taus}
Ibrahim found a missing cut in his monte carlo processing scripts, so now the Monte Carlo weights are working, and he has a set of EHE in MEOWS and he added it to GOLEM-fit, and are getting about 1.4 GZK events in MEOWS. Ibrahim has also been looking into some populations of EHE point sources that he has been propagating. 
We went through his draft slides: Slide 2, maybe change right to $E \times d\Phi/dE$ instead of just $d\Phi/dE$, Slide 4 maybe add Github repo, Slide 5 add author names, Slide 8 umlaut with Joran's name

Ibrahim will send around the slides once the event distributions are finished.

Carlos also mentioned a technical paper comparing UHE tau tools.