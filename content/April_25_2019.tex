\chapter{April 25, 2019}

\section{Plenary Talk Reviews}
\subsection{ANITA}
Ibrahim and I gave a practice ANITA plenary talk. We decided to not discuss the SNDAQ information, and to move the dead time discussion into the Followup sections, in addition to adding a plot with upper limits. A more comprehensive conclusion slides needs to be made, with mention of the application of TauRunner to other high energy problems.

\subsection{Gravitational Waves}
Justin and I think Raamis should emphasize that you perform a lot of background trials for the specific skymaps (emphasizing the statistics that you can gather with such a short latency).

Justin suggested asking for the realtime results from the Columbia group.

Justin's suggestion for a title tweak: \textit{Results from the archival and O3 Realtime Neutrino Follow up of Gravitational Waves}. He also suggests adding contours to GW localizations for the talk, not necessarily in realtime though because the marching triangles algorithm takes time.

I mentioned that another disadvantage of using the simulated skymaps for the Columbia analysis is that if there is any data/MC disagreement for skymaps on the LIGO end, that this will propagate into the Columbia analysis, but not Raamis's analysis. 

\section{ANITA regeneration paper}
Ibrahim also showed the paper outline for the regeneration paper. Francis had suggestions to declutter the plot comparing different injected neutrino energies with their secondary distributions to the analytical suppression. Carlos wants a vertical line at the injection energies, not bins, and then binning the smaller energies. I suggested moving the y limits lower to see the highest energy bin for the highest injected energy.

Justin asked for an increase in statistics in the exit probability jobs.

We also discussed the Eddington bias when comparing to the HESE flux. For that reason, we decided to use the HESE flux to set the maximum allowed flux that ANITA could have seen, not using the best fit of ANITA to look at how much it overshoots the HESE flux.

We decided to send an outline to PubCom today so that we can start working on the paper because we are worried that it could get scooped. 
