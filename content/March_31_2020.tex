

\chapter{March 31, 2020}

\section{O3 is dead}
\subsection{Long live O3}
Raamis will incorporate documentation for how to run GW code into an appendix of his thesis. I will work on bundling up latency stuff with Raamis's code so that it shouldn't be hard to turn back on. 

Also talked about hosting GW results on the ROC internal or external server. 

Also maybe having O4 cascades run in realtime even if Raamis graduates before then.

\section{AAS and APS planning}
Four contributions at AAS (Justin, Raamis, Abhishek, Me) and two for APS (Me and Justin).
For APS, Justin might want to update to the full O3 table of followups, for ANITA it should pretty much just be the ICRC slides.

Discussion later this week on whether or not we will send slides around to neutrino sources or just present because compiling GW plots could be a bit riskier.

\textbf{Neutrino 2020 planning}: Maybe GZK analysis if Ibrahim makes substantive progress in the next 2 weeks

On a serious note, talked about maybe submitting abstracts for everyone (ANITA, GW, FRA). Nothing definitive decided.

\section{$\tau$}
Ibrahim will be talking on the Diffuse call at 4:30 tomorrow about the secondary GZK analysis and discussing the bugs he unearthed. 

Also showed distributions in Truncated energy for both $\nu_{\tau}$ and $\nu_{\mu}$ for an injected $E^{-1}$, as well as comparisons to analytical expectations for effective area. Plot looks really promising, but Carlos mentioned some reasons for small discrepancies (geometric area vs. muon effective area, neutrino vs. average neutrino antineutrino).

Please look over Ibrahim's slides before tomorrow, specifically the zenith angle distribution plots. 

\section{Abhishek}
Abhishek compares the MOJAVE sample to some other samples used in other IceCube analyses. There's a decent amount of overlap between the IC86 blazar stacking analysis and Tessa's 10 yr point source. Only 2 overlapping with the ULIRG stacking, and only 56 from the RLAGN search (which has 13927 sources to start with).

For MOJAVE, there are memory problems when stacking many sources, I will help to take a look at this. 

Justin suggested a scatter plot for the common sources.

Justin and Ibrahim were a bit surprised at the low number of overlapping sources with MOJAVE and the RLAGN analysis (maybe because RLAGN is designed for off-axis?) 


\section{FRA}
Might be nice to see the background expectation too in addition to the ratio of 100 signal events when comparing to the alert event effective area. 