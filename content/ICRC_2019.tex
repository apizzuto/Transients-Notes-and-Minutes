\chapter{ICRC 2019}

\section{Day One (Thursday, July 25th)}
\subsection{Highlight and Review Talks}
\begin{itemize}
    \item \textbf{Opening Session: Welcome and Awards}: Albrecht's overview, hello from the Dean, NSF Director video, IUPAP overview, awards
    \item \textbf{Highlights from the Pierre Auger Observatory and prospects for AugerPrime (Antonella Castellina):} Began with an overview of the detector (water-cherenkov detectors with sub-array of dense stations for low energies, 4 fluorescence sites, underground muon detector, etc.). Improved measurement of the spectrum at lower energies ($10^{16.5}$ - $10^{18}$ eV). Assumes the evolution of the spectral slope is continuous, not broken. Now combined 5 spectra spectrum, agree well within systematics. Also measurements of mass, showing lighter spectrum up to around 2 EeV, heavier thereafter. Mass composition of the ankle: up to the ankle, exclude pure or p+He at over $6\sigma$. Statistical power now exists to help distinguish between astrophysical composition models. 3-D dipole above 8 EeV 6.6\%, amplitude seems to increase with energy. Most significant excess for $E > 38$ EeV about 2$^{\circ}$ from CenA. Cosmogenic limits maximum sensitivity at about and EeV. Upper limits for point source searches, no neutrinos found from BBH mergers. Goal for AugerPrime is not only to increase area, but also sensitivity to composition. \textit{Questions: How coupled are the significances of CenA and starburst galaxies?}
    \item \textbf{Non-Gamma-ray Applications of TeV Telescopes (Michael Daniel):} Probing non-TeV physics with TeV cherenkov telescopes by looking for physics in the EM band of cherenkov radiation (blue - UVish). ''Bigger is better'' in terms of photometric capability, so Cherenkov telescopes should be great for optical studies. Applications: Measure meteors up to 12th magnitude, setting limits on DM candidates (aggregates of up/down/strange quarks formed at early epochs, surrounded by electron shield), crab optical pulsar for timing, searching for optical emission from FRB 121102, asteroid size/shape imaging, fringe brightening from diffraction gives angular extent of stars. Intensity interferometry for sub-milliarcsecond astronomy. You can take measurements when it's not completely dark. Potentially able to count star spots with CTA (even just the MSTs).
    \item \textbf{Major Changes in Understanding of GRBs: Discovery of Teraelectron Volt Gamma-Ray Emission (Razmik Mirzoyan):} Mentioned first IACT detected GRB (GRB 190114C). Past hints from GRB 920925c from AIROBICC, but the signal was about a minute early and far from the source location, GRB 970417a and Milagrito. Around 8-10 GRB candidates per year at MAGIC. List of four different GRBs (1 short GRBs, 3 long). GRB160821B: sGRB, triggered by Swift-BAT, no LAT detection (kilonova?), 3.1$\sigma$ post-trials, energies from $\approx 600$ GeV - 800 GeV. Offline analysis over 50$\sigma$. Introducing a new unit, the kilo-Crab. HESS reported about excess from 180721B. 
\end{itemize}

\subsection{Parallel Talks}
\subsubsection{Neutrino Parallel (Convener: Aya Ishihara)}
\begin{itemize}
    \item \textbf{Multi-messenger interpretation of neutrinos from TXS 0506+056 (Walter Winter):} There is a qualitative difference between 2014-2015 flare and the high-energy emission, being cautious of low statistics Eddington Bias. Purely one zone leptonic models give no neutrinos, but purely one zone hadronic models overshoot x-ray and TeV gamma-ray data. More sophisticated geometric models might fit the data better. Different spectral shape if TXS is an FSRQ could potentially accomodate higher fluxes of neutrinos in the 2014-2015 period. 
    \item \textbf{Calorimetric Neutrino Expectations from Bright Blazar Flares  (Michael Kreter):} High-fluence blazars are best for individual source associations, Calorimetric output in BigBird field dominated by PKS B1424-418 (chance coincidence 5\%). Estimate maximum neutrino emission by assuming all gamma rays are hadronic, over-optimistic predicition helpful for intuition in model building. 3C 279 gave maximal number of neutrinos at IceCube of 0.02 (3C 279 and PKS 1510−089 dominate when you integrate over a sample of flares). Long-term association is calorimetrically (more) plausible (0.029 instead of $\mathcal{O}(10^{-5})$ for a short flare. \textit{Questions: Lots of questions about what's going on in radio, because you should still see them even if x-rays or gamma rays are suppressed}.
    \item \textbf{ The Pros and Cons of Beyond Standard Model Interpretations of ANITA Events (T. J. Weiler):} Summary of the anomalous ANITA events, listed some of the models to explain the ``1.5'' events. Francis got busted for revealing our analysis results when he was in Spain. Talk cut very short, but great list of references for ANITA papers
\end{itemize}
\subsubsection{Cosmic Ray Indirect (Convener: Pierre Sokolsky)}
\begin{itemize}
    \item \textbf{Cosmic Ray Extremely Distributed Observatory: Status and Perspectives of a Global Cosmic Ray Detection Framework (Dariusz Góra):} First approach at detecting large scale phenomena using cell phones, not necessarily tied to using cell phones. Spoke a lot about shower development near the Earth (presumably for an estimate of extent on the Earth?). Showed results of number of doublets with comparison to expectation. Mention future outlook: simulations of UHE photons, calibrating smartphones, correlation searches, gamification. \textit{Questions: somebody else asked a question about trusting the timing on cell phones, somebody accused the whole idea leading to obfuscation, not citizen science.}
    \item \textbf{The GRANDProto300 experiment} (I left to get to my poster)
\end{itemize}

\subsubsection{Neutrino Parallel Session 2 (Convener:  Tom Weiler)}
\begin{itemize}
    \item  \textbf{Determining the fraction of cosmic-ray protons at ultra-high energies with cosmogenic neutrinos (Arjen van Vliet):} Constraining upper limits on the cosmogenic neutrino flux can be used to place limits on the proton content of UHECR. If you play with the homogeneity of the source population of the rigidity-dependent cutoff, you can probe different cosmogenic neutrino scenarios. Combination of a large proton fraction and strong source evolution ruled out. This is independent of hadronic interaction models.
    \item \textbf{Measurement of the high-energy all-flavor neutrino-nucleon cross section with IceCube (Tianlu Yuan):} Described HESE sample, extends previous measurement to using the full HESE 7.5 year dataset with ternary PID. \textit{Question: What's the main systematic? Strong correlation between astrophysical flux parameters and cross-section parameters}
    \item\textbf{Fundamental Physics with High-Energy Astrophysical Neutrinos Today and in the Future (Mauricio Bustamante):} Lists the reasons why one should study fundamental physics with HE$\nu$. Lists the things that you can probe even in the face of astrophysical unknowns.
    \item \textbf{Flaring Rate Distribution of Gamma-Ray Blazars and Implications for High-Energy Neutrino Emission (Kenji Yoshida):} Latpop died, filling in the rest by memory a few hours later. Assumed similar calorimetric argument to look at how high fluxes from gamma-ray detected blazars could compare to the IceCube discovery potential. I think there was an error in how this was done, because it looked like the time-integrated discovery potential was rescaled by multiplying by a livetime ratio, but this would imply that we are sensitive to less time-integrated flux than we actually are. 
    \item \textbf{Expectations from the assumption of hadron-hadron collisions for high energy neutrinos (Carlo Mascaretti):} Implications on fitting for the prompt $\nu_e$ component in HESE, noting that the prompt $\nu_{e}$ component is more influential on the HESE sample than prompt $\nu_{\mu}$ is for through-going $\nu_{\mu}$
    \item \textbf{Multi-Messenger Connection among High-Energy Cosmic Particles (Kohta Murase):} Kohta gave a great talk, there was so much material though, so I'll just refer to his slides.
    \item \textbf{Characterizing the High Energy Activity of Blazars Possibly Correlated with Observed Astrophysical Neutrinos (Ankur Sharma):} No show
\end{itemize}

\section{Day Two (Friday, July 26th)}
\subsection{Highlight and Review Talks}
\begin{itemize}
    \item \textbf{A Brief History of Neutrino Astronomy (Francis Halzen):} A great talk on neutrino astronomy from Francis, not many notes because it's a lot of material that we know and I wanted to just listen.
    \item \textbf{Particle acceleration and where do ultra high-energy cosmic rays come from? (Katherine Blundell):} Studies from jet from a galactic microquasar allows for extremely precise physical measurements of the jet. dynamics. Summarized some magnetic instability mechanisms and what we can learn from observations as well as diffusive shock acceleration and the Hillas criterion. Extent is important, you need to access higher fourier modes, so a quick pitch for next gen radio interferometers. Jet ejecta and hotspots manifest DSA and particle acceleration but insufficient for UHECR (because of size, mainly). Backflow of radio galaxy lobes is promising (as well as relic giant radio lobes)
    \item \textbf{High-energy neutrinos from persistent and transient activities of compact objects (Ke Fang):} Neutrino production is the same as cooking a steak, you need the right ingredients, time to cook (interact), and time to set (decay primaries). Leptohadronic and hadronic models can lead to degeneracy in the neutrino fluxes, but polarization can be used to break the degeneracy. 
    \item \textbf{Highlights from the Telescope Array (Shoichi Ogio):} Detector overview (507 stations, 3 m$^2$ each, 1.2 km spacing, coverage $\approx$700 km$^2$. Event reconstruction motivated by empirical parameterizations. Energy spectrum from 11 years of data: fit with 3 power laws from right before the ankle to the highest energies. Different break point in the spectrum depending on the declination. Anisotropy analysis: hotspot, local significance 5.1$\sigma$, chance probability 2.9$\sigma$. Air shower time structure analysis updated. 
    \item \textbf{ Multi-Messenger Observations of GRBs: The GW connection (Elisabetta Bisaldi):} Began with brief introduction of GRBs (cosmological, large lorentz factor, prompt and afterglow phases, lGRBs from collapse of massive star and sGRBs from mergers, similar duration in spikes, SN connection, similar trends). All of these features come from observations in other bands. Listed detectors which contribute to the catalog of GRBs as well as introduced LIGO and next gen GW detectors. ``Transient GW signal'' signal with duration in the detector sensitive band significantly shorter than the observation time and that cannot be re-observed. Motivates low-latency sub-threshold analyses to boost low significances from either gamma-ray detectors or GW detectors. Went over some of the observables from GW150914 and GW170817/GRB170817A. Lists all of the science that joint detections get you (speed of gravity, equivalence principle, NS equation stat, hubble constant, central engine of GRBs, etc.)
\end{itemize}

\subsection{Parallel Sessions}

\subsubsection{Neutrino Parallel Session 3 (Convener:  Kohta Murase)}
\begin{itemize}
    \item \textbf{Search for neutrinos in IceCube from the local anisotropic universe using 2MRS (Steve Sclafani):} Typical analyses assume that the source of acceleration is also the beam dump. Steve chooses to correlate with density, hoping to find correlation with the beam dump if cosmic rays escape their acceleration environment. No correlation found. Limits set. 
    \item \textbf{ ANTARES search for point sources of neutrinos with 9 yr of data: a likelihood stacking analysis (Julien Aublin):} 8,000 events over 3125 days, visible sky from $+53^{\circ}$ and below. Stacking likelihood. There are a bunch of source catalogs (blazars, star forming galaxies, radiogalaxies, dust obscured AGN, and IceCube EHE and HESE events). Most significant is radiogalaxy 1.6$\sigma$ post-trial. Most significant sources: Radiogalaxy 3C403, Blazar MG3 J225517+2409 (there's also an EHE track there). There was a time-dependence test done that is like the opposite of a blind analysis
    \item \textbf{ Searching for High-Energy Neutrino Emission from TeV Pulsar Wind Nebulae (Qinrui Liu):} Analysis of 35 PWNe with gamma-ray emission above at least 1 TeV. Search is a stacking search. 
    \item \textbf{Recent Results from the Askaryan Radio Array (Amy Connolly):} Five stations are currently taking data in the ice, competetive sensitivity. 98\% of uptime is used in the analysis from 2013-2016 with new methods of dealing with noise. Triggers are efficient and keep about 90\% of neutrino events. Cuts in 2d space in SNR and cross-correlation. Mentioned the ARA Pulsers on strings 1 and 22 $\ldots$
    \item \textbf{Searches for Ultra-High-Energy Neutrinos with ANITA (Cosmin Deaconu):} I was way too nervous to take notes during this talk
    \item \textbf{A search for counterparts to ANITA neutrino candidates with IceCube (Alex Pizzuto):} 
\end{itemize}

\subsubsection{Neutrino Parallel Session 4 (Convener:  Subir Sarkar)}
\begin{itemize}
    \item \textbf{Measurement of the diffuse astrophysical muonneutrino spectrum with ten years of IceCube data (J\"oran Stettner):} Update to the diffuse $\nu_{\mu}$ analysis with more data. Forward folding fit, updated systematics, $\gamma = 2.28$, prompt fits to zero. If prompt, then $\gamma=2.24$.
    \item \textbf{Characterization of the Astrophysical Diffuse Neutrino Flux with High-Energy Starting Events and Prospects for Future Measurements with IceCube (Austin Schneider):} Austin gives a great and clear HESE talk as per usual. $\gamma = 2.9$, points out that the disagreement might not be an inconsistency, but a reflection of interesting physics (or just agreement within uncertainties). Room for spectral features from the different physics from sample to sample
    \item \textbf{Atmospheric Neutrinos Detected with the First KM3NeT Detection Units of ARCA and ORCA (Jannik Hofest{\"a}dt):} 50 days of ARCA2 data, 125 days of ORCA1 data. Good data-MC agreement for both muon and neutrino dominated regimes. Number of events: 77, expectation from $\nu$: 67.5, expectation from $\mu$: 4. Reconstruction of zenith angle of the muon is order a few degrees
    \item \textbf{Model Independent Unfolding of the Atmospheric Neutrino Event Rate by Volume in the 0.1-600 GeV Range (Joakim Sandroos):} Good explanation of unfolding vs. forward folding. Reasonable consistency in burn sample, some tension with expectation below 10 GeV and in up going region, and data release and publication in preparation, 
    \item  \textbf{Bounds on Diffuse and point source fluxes of ultra high energy neutrinos with the Pierre Auger Observatory (Francisco Pedreira):} Protons and nuclei make inclined showers high in atmosphere, and are mainly muons, neutrinos initiate ``deep'' showers close to ground and have an EM componenet. No neutrino candidate events found in downward-going and earth-skimming analysis. Limits can constrain some proton dominated cosmogenic models. Limits restrictive around 1 EeV
    \item \textbf{ The Baikal-GVD neutrino telescope: cascade events results (Rastislav Dvornick{\`y}):} 3 clusters of strings since 2018, about 860 OMs, almost double that by next year, effective volune 0.15 km$^3$. There are 3 cascade events with energy over $100$ TeV (1.4 astrophysical expected). Upper limit on diffuse neutrino flux assuming $E^{-2.46}$
\end{itemize}

\subsubsection{Discussions that I don't want to forget about}
\begin{itemize}
    \item \textbf{Mary Hall Reno, before ANITA talk:} We discussed a bit about our calculation (just what was going to be said in the talk). She mentioned that she wants her code to be open source, it will include EAS evolution in the atmosphere, and is written in FORTRAN. She is happy to hear that our results agree roughly with her plots of $\tau$ exit probability as a function of $\beta_{tr}$. She points out that her work is different than \texttt{NuTauSim}, but she compares against their results
    \item \textbf{Fabian at FRB poster:} Discussed FRB models, just wanted to introduce myself
    \item \textbf{Romero-Wolf, after ANITA talk:} Introduced himself, he mentioned that he was working on extending his ANITA acceptance paper to include point source hypotheses
    \item \textbf{One of Aloisio's students, can't remember name, at poster session:} He wants to look at BSM models that would predict EHE $\tau$ tracks in IceCube. Is this still happening with Derek's student?
    \item \textbf{ANTARES transient folks (Damien, Mukharbek, etc.):} weird conversation about the FRB analyses, didn't really understand everything. One of them asked why the ANTARES limits aren't included on the limit plot.
\end{itemize}


\section{Day Three (Saturday, July 27th)}
\subsection{Highlight and Review Talks}
\begin{itemize}
    \item \textbf{Thermal WIMPs on the Brink (Tim Linden):} Explains motivations in data for thermal WIMPs because of excesses in particle-antiparticle ratios. Most models of dark matter that rely on positron excesses are excluded by Planck bounds, but AMS data suggests an anti-proton excess which could be consistent with dark matter. For AntiNuclei (anti-deuteron, anti-helium, etc.) low energies are essentially astrophysically background free. Potential detection of a few anti-helium detections at AMS. Looking at gamma rays, galactic center excess (bright, hard-spectrum, spherically symmetric, spatially extended). But there are also a lot of SM explanations (pulsars, mainly), but evidence suggesting that it might be pulsars might be biased to prefering point sources (not DM). Radio (SKA) should definitively find the pulsars. Dwarf spheroidal galaxies: very dim, DM content known from optical data (analyzing with Fermi-LAT data is systematically limited). At higher energies, HESS can actually cut into DM masses below the thermal cross-section at 1 TeV, and CTA will be able to do even better. At lower energies, next gen radio and also EDGES are / will be constraining
    \item \textbf{Exploring the Extreme Universe with Gamma-ray Observatories (Rashmi Mukherjee):} Review of the relevant energy ranges and detection techniques of satellites and IACTs. Shows the Fermi gamma-ray sky and the new Fermi catalog, some results from HAWC (galactic plane map) and studies with IACTs of specific objects (Crab, etc.)
    \item \textbf{ The CALorimetric Electron Telescope (CALET) on the International Space Station (Yoichi Asaoka):} Overview of the detector and components (charge detector, imaging calorimeter, total absorption calorimeter). Showed extended measurement of all electron spectrum, cosmic-ray proton spectrum, gamma-ray observations (search for GW counterparts), electron anisotropy.
    \item \textbf{Recent Results of Cosmic Ray Measurements from IceCube and IceTop (Dennis Soldin):} Cosmic ray spectrum measured with 3 years of data and $5\times10^7$ events, compared to new results with one year of data from 2016-2017, agrees well in overlap region
    \item \textbf{ Latest Results from the Alpha Magnetic Spectrometer on the International Space Station (Bruna Bertucci):} 
\end{itemize}

\subsubsection{Neutrino Parallel Session 5 (Convener:   Enrique Zas)}
\begin{itemize}
    \item \textbf{ Search for Neutrino Emission in IceCube's Archival Data from the Direction of IceCube Alert Events (Martina Karl):} Search for time-integrated emission in point-source tracks data consistent with the PSFs from historic IceCube Alert events. Somebody asked why the full likelihood information of the alert event isn't used 
    \item \textbf{ ANTARES-IceCube Combined Search for Neutrino Point Sources in the Southern Hemisphere (Giulia Illuminati):} We've heard about this analysis a lot
    \item \textbf{All Sky Time-Integrated Point Source Searches using 10 years of IceCube Data (Tessa Carver):} We've heard about this analysis a lot
    \item \textbf{ Search for High-energy Neutrinos from AGN Cores (Federica Bradascio):} We've heard about this analysis a lot \textit{Question about if the orientation is taken into account, because the orientation can affect the attentuation of the x-ray flux severely (and NGC 1068 is actually an example of this)}.
    \item \textbf{ Search for Astronomical Neutrino from Blazar TXS0506+056 in Super-Kamiokande (Kaito Hagiwara):} 68\% of all events included in the ``search cone'' (5 - 10 degrees). No significant signal was found in a time-integrated test. KS test performed for time-dependence, no significant clustering.
    \item \textbf{Searching for Time-Dependent Neutrino Emission from Blazars (Erin O'Sullivan):} We've heard about this analysis a lot
\end{itemize}
SNR analysis using GRECO?

\subsubsection{Neutrino Parallel Session 5 (Convener: Subir Sarkar)}
\begin{itemize}
    \item \textbf{ Multi-messenger Gravitational-Wave + High-Energy Neutrino Searches with LIGO, Virgo and IceCube (Azadeh Keivani):} Bayesian approach instead of typical IceCube Frequentist approach for a point source analysis. In principle sensitive to sub-threshold events from LIGO. IceCube has been the fastest to get GCN circulars out!
    \item \textbf{ IceCube Search for Galactic Neutrino Sources based on HAWC Observations of the Galactic Plane (Ali Kheirandish):} Galactic plane is a rich environment to accelerate and interact cosmic rays. If sources are hadronic, TeV $\gamma$ rays should be accompanied by neutrinos with similar energies. HAWC sees northern sky, where IceCube is most sensitive, and operates at high energies. Limits are sooooo close to constraining the hadronic component of the $\gamma$-ray flux. Template analysis can constrain the hadronic component as well as zooming in on the Cygnus region (at the level of 60\%).
    \item \textbf{Search for Correlations of High-energy Neutrinos and Ultra-high Energy Cosmic Rays (Anastasia Barbano):} 3 different analyses, some are updates to previous analyses.
    \item \textbf{ ANTARES 2007-2017 Search for Point Sources Using All Neutrino Flavours (Sergio Navas):} 11 years of data, unbinned likelihood analyses. Assume $E^{-2}$. Time-independent and time-dependent analyses performed. No significant signal found. EHE event at 343, 23 corresponds to the most significant cluster in ANTARES data (2.4$\sigma$ post-trials). Public data release
    \item \textbf{ Probing neutrino emission at GeV energies from compact binary mergers with IceCube (Gwenha{\"e}l De Wasseige):} Gwen talked about her GeV search in coincidence with GWs. No significant signal, but a really cool analysis. 
    \item \textbf{A DECam Search for Explosive Optical Transients Associated with IceCube Neutrinos (Robert Morgan):} 
    \item \textbf{Observation of Optical Transients and Search for PeVEeV Tau Neutrinos with Ashra-1 (Satoru Ogawa):}
    \item \textbf{ ANTARES search for high-energy neutrinos from TeVemitting blazars, Markarian 421 and 501, in coincidence with HAWC gamma-ray flares (Mukharbek Organokov):}
\end{itemize}