\chapter{GRECO Astronomy Call: November 1, 2019}

\subsection{Search for Transient Sources with GRECO: Chujie}
Chujie presents an update on his GRECO transient analysis using \texttt{cSky}. Previous implementation used constant 20$^{\circ}$ error, but this is underestimated. New method is similar to Mia's, using a 2d spline of logE and $\sin(\delta)$, but with bins of equal solid angle, not equal declination. Northern sky events tend to have smaller errors, median angular error gets smaller for higher energies ($>$250 GeV). Chujie shows some odd behavior in background TS distributions, southern sky locations always have TS of 0, horizon analyses have TS peaked at non-zero values. 
\textit{Questions} Michael: Have you looked at event rates to make sure that there are events for the southern declinations? Alex: I saw similar behavior when assigning large angular uncertainties, can you look at distributions of assigned angular uncertainties in slices of $\sin(\delta)$? It might help to look at 1D projections of the splines in energy for different declination bins. 

\subsection{Novae Update: Alex}
I gave an update about the random forest I'm using to estimate angular uncertainties. Extra dimensionality means that you can assign different uncertainties even for events in the same declination and reconstructed energy bins. Ignacio asked if it might be easier to just train on reco muon direction to true muon direction (instead of neutrino truth), and then use the new PS methods to do the pull-correction equivalent. I said I can reach out to Hans about this. Showed random forest model performance, seems to do well enough to include in a likelihood. Michael suggested making the histogram on slide 5 for different slices of true and reconstructed energy. I also showed ORCA comparisons. Michael mentions that there are cuts applied after the NMO sample so the Eiso plot might change. We all discussed trying to streamline fluxes and what we are showing with someone from ORCA, and then maybe trying to prepare a comparison for a 2020 conference like Neutrino2020.

\subsection{Angular Uncertainties: Michael}
Michael gives a reminder about the MultiNest procedure and mentions some problems with just adding up Kent distributions from the ellipses (mainly correlation). Michael shows skymaps from current method (going from ellipses to fitting a von-mises to fitting a Kent to fitting a \textit{second} Kent). Open questions: how do we pull correct when there are two sigma values? Suggestions are welcome. Ignacio: Can you show what one of these maps looks like when azimuth is not strongly constrained. Michael working on processing all of these, and then they can be used in likelihood analyses with a 10 parameter spatial distribution that is analytic