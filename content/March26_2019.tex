\chapter{March 26, 2019}

\section{FRB Paper}
Sam is in Minnesota and Ali is still out of town. Sam is forwarding me the list of comments, and we will discuss who can respond to what comments. 

\section{ANITA}
Ibrahim and I have been asking around about muongun simulation. First, we were directed to HESE simulation, but all the muongun was downgoing (no simulation below the horizon). The lead we are currently investigating is from a diffuse analysis done in 2012. There is muongun simulation processed to l1, l2, and l3. It is poorly documented, so I am looking at the files to try to determine the underlying simulation distributions, and Ibrahim is figuring out if simulation from 2012 (ie before pass2) will work with Tessa's event selection scripts. 

\section{Gravitational Waves}
Raamis is working with Erik on GCN templates, examples in his Wiki. As for reporting upper limits, Raamis has been working on calculating the standard point source sensitivity (no prior) for 5degree bins in declination across the sky, and then the best and worst for within the 90\% containment could be cited.

We also spoke a lot about the conversation on fitting for underfluctuations, specifically in limit setting. Ibrahim advocates for using underflucuations for more constraining upper limits. On the call it was mentioned that Raamis might be able to show that he could also set more constraining limits if he were to fit for underfluctuations, but I raised a potential problem with this (there is essentially no way to get a negative TS value based on the way that the spatial prior analysis works, because the pixel with the highest LIGO weight is not penalized at all, so that pixel is guaranteed to have a TS$\geq 0$).

Raamis sent an email to Scott asking for GCN circular submission credentials, and I am helping with writing a script to call people to wake us up if a LIGO alert comes overnight. 

\section{Collaboration Meeting Planning}
Briefly discussed planning for the collaboration meeting. Raamis is hopeful that there will be some new LIGO events to discuss during a Nu-Sources or Realtime parallel, and I am hoping to finish the grbllh/Skylab Fast Response conversion and present that in Realtime. We are also open to more talks if anyone has suggestions. 


